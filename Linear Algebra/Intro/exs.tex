\documentclass{article}
\usepackage{amsmath}
\usepackage{amssymb}
\usepackage{amsfonts}

\newcommand{\qst}[1]{\begin{center}\textit{\textbf{Problem #1}}\end{center}}
\newcommand{\sol}{\begin{center}\textit{\textbf{Solution}}\end{center}}
\newcommand{\thf}{\\\therefore}
\newcommand{\dx}{\hspace{1mm}dx}
\newcommand{\du}{\hspace{1mm}du}
\newcommand{\dy}{\hspace{1mm}dy}
\newcommand{\tit}[1]{\textit{#1}}
\newcommand{\qed}{\\\begin{center}$\therefore \textbf{\textit{Q.E.D.}}$\end{center}}
\newcommand{\pt}[1]{\\\begin{center}\textit{\textbf{Part #1}}\end{center}}
\newcommand{\ptf}[1]{\begin{center}\textit{\textbf{Part #1}}\end{center}}
\newcommand{\hs}[1]{\hspace{#1mm}}
\newcommand{\subsec}[1]{\newpage \subsection*{#1}}
\title{Introduction to Linear Algebra Excersises}
\author{inductionstove}
\date{2021}

\begin{document}
\maketitle
\section*{Chapter 1}
\subsection*{Exercise 3}
\begin{center}\textit{\textbf{Problem 3}}\end{center}
Given that $A$ and $B$ are two vectors defined on the same dimentions, prove that $(A+B)^2=A^2+B^2+2AB$, and that $(A-B)^2=A^2+B^2-2AB$
\sol
\begin{gather*}
A=(a_1, a_2, a_3,....a_n)
\thf A^2 = (a_1^2, a_2^2, a_3^2,....a_n^2)
\\B=(b_1, b_2, b_3,....b_n)
\thf B^2=(b_1^2, b_2^2, b_3^2,....b_n^2)
\thf 2AB = (2a_1b_1, 2a_2b_2, 2a_nb_n)
\thf (A+B)=(a_1+b_1, a_2+b_2.....a_n+b_n)
\thf (A+B)^2=((a_1^2+b_1^2+2a_1b_1),(a_2^2+b_2^2+2a_2b_2),...(a_n^2+b_n^2+2a_nb_n))
\\=(a_1^2, a_2^2, a_3^2,....a_n^2)+(b_1^2, b_2^2, b_3^2,....b_n^2)+(2a_1b_1, 2a_2b_2, 2a_nb_n)
\\=A^2+B^2+2AB
\end{gather*}
\qed
PTO
\\Now, we move on to the next proof.
\begin{gather*}
A=(a_1, a_2, a_3,....a_n)
\thf A^2 = (a_1^2, a_2^2, a_3^2,....a_n^2)
\\B=(b_1, b_2, b_3,....b_n)
\thf B^2=(b_1^2, b_2^2, b_3^2,....b_n^2)
\thf 2AB = (2a_1b_1, 2a_2b_2, 2a_nb_n)
\thf (A-B)=(a_1-b_1, a_2-b_2.....a_n-b_n)
\thf (A-B)^2=((a_1-b_1)^2, (a_2-b_2)^2,....(a_n-b_n)^2)
\\=((a_1^2+b_1^2-2a_1b_1),(a_2^2+b_2^2-2a_2b_2),...(a_n^2+b_n^2-2a_nb_n))
\\=(a_1^2, a_2^2, a_3^2,....a_n^2)+(b_1^2, b_2^2, b_3^2,....b_n^2)-(2a_1b_1, 2a_2b_2, 2a_nb_n)
\\=A^2+B^2-2AB
\end{gather*}
\qed
\qst{5}
Given that vector $A$ is perpendicular to any and all vectors, prove that $A$ is a 0 vector. You may take any and all vectors to be $X$.
\sol
\begin{gather*}
A=(a_1, a_2, a_3,....a_n)
\\X=(x_1, x_2, x_3,....x_n)
\\AX=\sum^{n}_{i=1}a_ix_i
\end{gather*}
Because $x_i$ can be any number, the only way the sum of all the terms is 0 is by ensuring that all the terms are equal to 0, and that can only be the case if $a_i$ is 0 in all cases, because 0 is the only number which yieds 0 as the product when multiplied with any number.
\begin{gather*}
\therefore a_i = 0 \text{ for all }i.
\thf A \text{ is a 0 vector.}
\end{gather*}
\subsec{Exercise 4}

Find the norm of the vector A in the following cases.
\pt{A}
A=(2, -1)
\sol
\begin{gather*}
||A||=\sqrt{\sum^{n}_{i=1} a_i^2}
\\=\sqrt{2^2+(-1)^2}=\sqrt{4+1}=\sqrt{5}
\end{gather*}
\pt{E}
$$A=(\pi, 3, -1)$$
\sol
\begin{gather*}
||A||=\sqrt{\sum^{n}_{i=1} a_i^2}
\\=\sqrt{\pi^2+3^2+(-1^2)}
\\=\sqrt{\pi^2+10}
\end{gather*}
\qst{3}
Find the projection of $A$ along $B$.
\pt{B}
$$A=(-1, 3);\hs{1} B=(0,4)$$
\sol
\begin{gather*}
c=\frac{A\cdot B}{B\cdot B}
\\=\frac{(-1\cdot 0)+(3\cdot 4)}{(0\cdot0)+(4\cdot4)}=\frac{12}{16}=\frac{3}{4}
\\\text{Projection of A along B }=cB
\\=\frac{3}{4}(0,4)=(0,3)
\end{gather*}
\qst{5}
Find the cosine between the following vectors $A$ and $B$.
\ptf{A}
$$A=(1, -2);\hs{1} B=(5, 3)$$
\sol
\begin{gather*}
A\cdot B=1\cdot5+-2\cdot3=-1
\\||A||=\sqrt{1^2+(-2)^2}=\sqrt{5}
\\||B||=\sqrt{5^2+3^2}=\sqrt{34}
\\\cos{\theta}=\frac{A\cdot B}{||A||\hs{0.5}||B||}
\\=-\frac{1}{\sqrt{5}\sqrt{34}}
\end{gather*}
\qst{6 A}
Find the cosine of the angle formed by these vertices:
$$(2, -1, 1),\hs{1}(1, -3, 5), \hs{1}(3, -4, -4)$$
\sol
I shall consider the angle to be denoted by $\vartheta$ instead of $\theta$ to phleks the seski math symbol.
\\Firstly, we shall lable the points.
\begin{gather*}
\\(2, -1, 1) = A
\\(1, -3, 5) = B
\\(3, -4, -4) = C
\\\overrightarrow{AB}=B-A=(-1, -2, 4)
\\\overrightarrow{BC}=C-B=(2, -1, -9)
\thf \cos{\vartheta} = \frac{A\cdot B}{||A||\hs{0.5}||B||}
\\=\frac{(-1)\cdot2+(-2)\cdot(-1)+4\cdot9}{(\sqrt{(-1)^2+(-2)^2+4^2})(\sqrt{2^2+(-1)^2+(-9)^2})}
\\=\frac{36}{\sqrt{21}\sqrt{41}}
\\
\\f'(x)=f'(g(x))\cdot g'(x)
\end{gather*}
\qst{7}Actually fuck it if you wanna see the solution head to my chat with $yang\_neo$ so ya.
\\No wait we chat about a lot of shit which you dont wanna see youre gonna judge us so dont
\newpage
\subsec{Exercise 6}
\qst{2}
Let  $y  =  mx + b$ and  $y  =  m' x  + c$  be  the  equations  of  two  lines  in  the  plane. Write  down  vectors  perpendicular  to  these  lines. Show  that  these  vectors  are perpendicular  to each  other  if  and  only  if  $mm'  = - 1$
\sol
\begin{gather*}
b=y-mx
\\c=y-m'x
\thf (1, -m)= \text{Vector }\beta \{\text{seski phlex}\},
\\(1, -m')= \text{Vector }\gamma \{\text{seski phlex again}\}
\\\beta \cdot \gamma = 0
\thf 1\cdot 1 + -m\cdot(-m')=0
\thf 1+mm'=0
\thf mm'=(-1)
\end{gather*}
\qed
\end{document}