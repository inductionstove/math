\documentclass{article}
\usepackage{amsmath}
\usepackage{amssymb}
\usepackage{amsfonts}
\usepackage[utf8]{inputenc}

\newcommand{\qst}[1]{\begin{center}\textit{Problem #1}\end{center}}


\title{Speedrun of Exercises of Chapter 11}
\date{June 2021}
\author{Sammit Ramanan}

\begin{document}

\maketitle

\section*{11.2}
\qst{1. A}
A sequence is an ordered set of numbers.
\\A series is the summation of a sequence.
\qst{1. B}
A convergent series is one where the sum of the series exists as a finite number.
\\A divergent series is one where the sum of the series exists as a finite number.
\qst{2}
It would mean that the sum of the infinite series $a_n$ is five.
\qst{3}
I saw the answer so cant really do much now.
\qst{4}
\begin{gather*}
s_n=\frac{n^2-1}{4n^2+1}
\\\therefore \lim_{n\to\infty}s_n=\frac{1}{4}
\end{gather*}
\qst{5}
1, 0.125, 0.037
\\Looks like it is convergent.
\\Imma not do 6-8 because main tera baap ka naukar nahi hun.
\qst{9}
pukka convergent hai main guarantee deta huu
\qst{49}
x=1
\\\begin{gather*}
\frac{9}{10^n}=s_n
\\\therefore a=\frac{9}{10}, r=\frac{1}{10}
\\\therefore s_{\infty}=\frac{\frac{9}{10}}{1-\frac{1}{10}}
\end{gather*}
\end{document}