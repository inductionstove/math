\documentclass{article}
\usepackage[utf8]{inputenc}
\usepackage{amsmath}
\usepackage{amssymb}
\title{I will change it stfu}
\author{r.suhrit2017}
\date{March 2021}
\begin{document}
\maketitle
\section{Introduction}
Unless specifically mentioned, nth term will always indicate nth term of the sequence $b_n$.
\\\\1
\\a. An alternating  series is one where the sign of the term alternates, i.e., when every term of a sequence is multiplied with $(-1)^{n}$ or $(-1)^{(n+1)}$.
\\b. If an alternating seies is constantly decreasing, and the terms of the series approach zero as $n$ becomes arbiterarily large, the series is convergent.
\\c. This shows that the remainder of the partial sums is leser than the next term.
\\\\2. $$\sum^{\infty}_{n=1}\frac{2(-1)^{n-1}}{2n+1}$$
\\\begin{gather*}
n^{th} \text{ term}=\frac{2}{2n+1}
\\\text{The first condition is clearly satisfied. We test for the second condition:}
\\\lim_{n\to\infty}\frac{2}{2n+1}=\lim_{n\to\infty}\frac{\frac{2}{n}}{2+\frac{1}{n}}=\frac{0}{2}=0
\end{gather*}
With both the conditions satisfied, we can conclude that the series is convergent.
\\\\3. $$\sum^{\infty}_{n=1}\frac{2n(-1)^{n-1}}{n+4}$$
\\We find that the sequence is increasing, for the nth term =$$\frac{2n}{n+4}$$ Through calculus it is obvious that the rate of increment in the numerator is higher, and thus the sequence starts to increase after the first 4 terms. This proves that it is not strictly increasing, and so the sequrnce is divergent.
\\\\4. $$\sum^{\infty}_{n=1}\frac{(-1)^{n-1}}{\sqrt{n}}$$
\\The alternating sequence $$\frac{1}{\sqrt{n}}$$ is clearly convergent what the fuck is wrong with you its clearly decreasing and clearly approaching zero as n approaches infinity.
\\\\5. $$\sum^{\infty}_{n=1}\frac{(-1)^{n-1}}{\ln{n}+4}$$
The series $$\sum^{\infty}_{n=1}\frac{1}{\ln{n}+4}$$ satisfies $b_{n+1}<b_{n}$ because $\ln{n}$ is a strictly increasing function. Then, we have
\begin{gather*}
\lim_{n\to\infty}\frac{1}{\ln{n}+4}
\\=\frac{1}{\infty+4}=\frac{1}{\infty}=0
\end{gather*}
\\Thus we have proved that the series is convergent.
\\\\8. $$(-1)^{n-1}\frac{n}{\sqrt{n^3+2}}$$
We will divide this but the highest power in the nummerator by the highest in the denominator:
\begin{gather*}
\text{Numerator:}
\\\frac{n}{n^{\frac{3}{2}}}=n^{1-\frac{3}{2}}=\frac{1}{\sqrt{n}}
\\\text{Dinominator:}
\\\frac{\sqrt{n^3+2}}{n^{\frac{3}{2}}}=\sqrt{\frac{n^3}{n^3}+\frac{2}{n^3}}
\\\text{Now we take the fraction as a whole:}t
\\\frac{\frac{1}{\sqrt{n}}}{\sqrt{1+\frac{2}{n^3}}}=\frac{1}{\sqrt{n+\frac{2}{n^2}}}
\\\lim_{n\to\infty}\frac{1}{\sqrt{n+\frac{2}{n^2}}}=0
\end{gather*}
The second condition asks us to prove that $b_n$ is strictly decreasing. This is true, and one can confirm it by graphing the function. Even though the function $b_x$ is not strictly decreasing, if the domain is restricted to natural numbers then $b_{n+1}\geq b_n$, and thus it can be concluded that the series is convergent.
\\
\\23
\\Firlty we have to prove that the seires is convergent. It has alreay been seen that the zeta function is convergent with the domain is restricted in this way: $$\{x: x>1\}$$
Yet, we will look at a specific proof because bo class is damn boring.
\begin{gather*}
\int^{\infty}_{1} \frac{1}{n^6}\hspace{1mm}dx=\lim_{t\to\infty}int^{t}_{1} \frac{1}{n^6}\hspace{1mm}dx=\lim_{t\to\infty} -\frac{t^{-5}}{5}+\frac{1}{5}=\lim_{t\to\infty} \frac{-1}{5t^{5}}+\frac{1}{5}=0+\frac{1}{5}=\frac{1}{5}
\end{gather*}
\\Because 0.2 is a finite number the integral is convergent, and so is the series.
\\For error less than 0.00005;
\begin{gather*}
\\0.00005\leq\frac{1}{n^6}
\\\therefore 0.00005\leq n^{6}
\\\therefore \sqrt[6]{\frac{1}{0.00005}}\sqrt[6]{\frac{100000}{5}}=\sqrt[6]{20000}\approx 5.21\leq n\sqrt[6]{\frac{100000}{5}}=\sqrt[6]{20000}\approx 5.21\text{, so we can take the value }n=5
\\\therefore \text{For an approximation with error 0.00005, we can take }n+1=5
\\\therefore n=4.
\end{gather*}
\\\\24
\\Before proceeding, one must observe that the fraction is multiplied by $(-1)^n$ instead of $(-1)^{n-1}$. Since our formula only works for $(-1)^{n-1}b_n$, we can multiply divide the fraction with -1, resultint in a - sign appearing in front of the faction.
\\If error less than 0.0001 then $b_{n+1}$ >
 0.001.
\\Solving the inequality:
\begin{gather*}
\frac{1}{n5^n}=0.0001
\\\therefore n5^n=10000
\\\therefore n=\frac{2^4\cdot 5^4}{5^n}
\\\therefore n=16\cdot 5^{4-n}
\\\therefore 10000=(16\cdot 5^{4-n})5^n
\end{gather*}
\\
\\36.
\\a.
\\Let us split the sums of $s_{2n}$ and $h_{2n}$ into the sums of their odd and even terms. Since the sums of their odd terms are equal anyway, we can subtract them from both sides of the equation. A new variable, $a$ shall be used. This indicates the nth term but since we have the limits as $n$ and $2n$, $a$ is used for clarity.
\begin{gather*}
\sum^{2n}_{a=1}\frac{(-1)^{2a-1}}{a}=\sum^{n}_{a=1}\frac{1}{2a}-\frac{1}{a}
\\\therefore \frac{-1}{2a}=\frac{-1}{2a}
\end{gather*}
Both the sides of the equation now mathc, hence proved.
\\b.
\\
\begin{gather*}
\sum^{2n}_{n=1}\frac{(-1)^{n-1}}{n}+\sum^{n}_{n\to 1}\frac{1}{n}-\ln{2n}\to\gamma
\\\text{as }n\to\infty
\\\therefore \lim_{n\to\infty}\sum^{2n}_{n=1}\frac{(-1)^{n-1}}{n}+\sum^{n}_{n\to 1}\frac{1}{n}-\ln{2n}=\lim_{n\to\infty}\sum^{2n}_{n\to 1}\frac{1}{n}-\sum^{2n}_{n=1}\frac{(-1)^{n-1}}{n}-\ln{n}
\\\therefore \lim_{n\to\infty}\sum^{2n}_{n=1}\frac{(-1)^{n-1}}{n}+ \lim_{n\to\infty}\sum^{n}_{n\to 1}\frac{1}{n}- \lim_{n\to\infty}\ln{2n}=\lim_{n\to\infty}\sum^{2n}_{n\to 1}\frac{1}{n}- \lim_{n\to\infty}\sum^{2n}_{n=1}\frac{(-1)^{n-1}}{n}- \lim_{n\to\infty}\ln{n}
\\\therefore  \lim_{n\to\infty}\sum^{2n}_{n=1}\frac{(-1)^{n-1}}{n}=\lim_{n\to\infty}\ln{2n}-\ln{n}=\ln(\frac{2x}{x})
\\\therefore \lim_{n\to\infty}\sum^{2n}_{n=1}\frac{(-1)^{n-1}}{n}=\ln{2}
\\\therefore \sum^{\infty}_{n=1}\frac{(-1)^{n-1}}{n}=\ln{2}
\\\therefore \textbf{\textit{QDL}}
\end{gather*}
\end{document}




















