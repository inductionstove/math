\documentclass{article}
\usepackage[utf8]{inputenc}
\usepackage{amsmath}
\usepackage{amssymb}
\title{I will change it stfu}
\author{r.suhrit2017 }
\date{March 2021}

\begin{document}

\maketitle

\section{Introduction}
To prove:$$\frac{n!}{n^n}\leq \frac{1}{n}$$
This will be equal to:
\begin{gather}
\frac{n!}{n^n}=\frac{1\cdot 2\cdot 3 .... \cdot n}{n\cdot n ...\cdot n}
\end{gather}
Cancelling the one $n$ from the denominator and another form the numerator and taking the $1/n$ separately, and separating the fractions we have:
$$\frac{1}{n}\frac{2\cdot 3\cdot .... \cdot (n)}{n^{(n)}}$$
By definition, we have the same number of integers to be multiplied in the numertor and the denominator. This also means that except the last term every integer is less than $n$, creating a number less than 1 (Unless $n=1$). Multipliying any number by another which is less than one produces a lesser absolute value (here we are dealing with +ve integers anyway.), and multiphying with one brings the same value.
\\$\therefore$ Proved.

\section{Question 92}
a.
\\Solution; It is given that $a_{2n+1}$ approaches $L$ as $n$ becomes arbiterarily large. This means that all terms mapped to an even value of $n$ approach $L$. What remains is odd numbers, for which simoilar data has been given. This only proves that the values of a sequence as a whole approaches $L$, a number. Thus the sequence is convergent.
\\b.
\\Solution:
\begin{gather*}
\lim_{n\to\infty}a_{n+1}=1+\frac{1}{1+a_{n}}
\end{gather*}
Here, we have to find out the value that the sequence is approaching, and for that we can substitute $n=\infty$.
\begin{gather*}
\lim_{n\to\infty}a_{n+1}=1+\frac{1}{1+a_{n}}=a_{n}
\\\therefore (a_n+1)(a_n-1)=1
\\\therefore ({a_{n}})^2-1=1
\\\therefore ({a_{n}})^2=2
\\\therefore {a_{n}}=\pm \sqrt{2}
\end{gather*}
But seeing that the sequence is strictly increasing after calculating the values of $n=1,2,3,4,5,6,7,8$ we can conclude that the value cannot be negative, and so the limit of the sequence is the positive square root of 2.
\end{document}